\documentclass[a4paper]{article}

\usepackage[english]{babel}
\usepackage[utf8]{inputenc}
\usepackage{hyperref}

\usepackage[backend=biber, bibencoding=utf8]{biblatex}
\bibliography{literature.bib}

\title{Thesis Proposal: Conditioning of DDPMs on Accelerated MRI}
\author{Lionel Peer}

\date{\today}

\begin{document}
\maketitle
\vfill
\begin{tabular}{l l}
    Start       & 18.09.2023                               \\
    End         & 25.12.2023                               \\
    Supervision & \textbf{Prof. Dr. Ender Konukoglu}       \\
                & Institute for Biomedical Image Computing \\
    Advisers    & Georg Brunner \& Emiljo Mehillaj
\end{tabular}
\tableofcontents
\newpage

\section{Background}
Data acquisition in magnetic resonance imaging (MRI) takes a long time and reducing this acquisition time has been a long standing research problem for the following reasons:
\begin{enumerate}
    \item MRI machines could perform more scans, driving down the cost per patient, and opening up diagnosis with MRI for a larger number of patients.
    \item Better performance on dynamic imaging, since the temporal resolution could be increased.
    \item Higher patient comfort and less unsuccessful scans due to patient motion.
\end{enumerate}
Recently, methods using undersampling of Fourier space have received much attention and with the rise of generative deep learning~\cite{kingma2022autoencoding}
\section{The Thesis Project}
\section{Project Schedule}
\section{Supervisor and Support}

\printbibliography[title=References]
\end{document}